\chapter*{Abstract}

%According to the guidelines of the faculty, an abstract in English has to be inserted here.
% Since the dawn of computer graphics natural phenomena have served as a permanent topic of interest for researchers and enthusiasts.
% To reproduce nature's elegance and beauty has proven to be a task of immense complexity. Skylight, clouds, a sunset. Water in various environments.
% Glass of water generating caustics. A child exploring fluid dynamics in the bathtub. A lake at rest, reflecting the surrounding mountains.
% The endless ocean bearing detail on every scale: the foam and spray caused by breaking waves, the glimmer on the ocean surface caused by waves
% with wavelengths smaller than a centimeter, the subtle shift in water color depending on the local underwater flora and fauna.

The synthesis of a believable depiction of the ocean surface is a permanent
topic of interest in computer graphics. It represents even more of a
challenge for applications which require the real-time display of a water
body as large as the ocean.
That is because the ocean is a highly dynamic system which combines waves
at all scales, ranging from millimetres to kilometres. Moreover, the
ocean may be observed from several distances, ranging from close-ups
to views which reach the horizon. 
%To complicate matters further, the
%ocean is illuminated by natural light sources, such as the sun and the
%sky, which are dynamic by nature too.
Thus, we present a framework to generate and render the open ocean in
real time, for arbitrary viewing distances and including waves at all
scales. We focus our efforts on the geometry of the animated
ocean surface, for which we leverage a set of wave spectrum models from
oceanographic research. We discuss the intricacies of
said models, as well as their fitness for real-time rendering. Moreover,
we delineate in detail how to integrate distinct wave spectrum models into a
coherent framework, from which one is able obtain believable, consistent
and coherent results.

%The scope of this thesis includes the generation, animation and rendering of the
%surface of an open ocean in real-time. We focus our interest on the synthesis of
%animated ocean surface geometry, for which we will adopt a set of models from
%oceanographic research. Specific properties of said models allow for
%easy addition and reduction of detail, as well as for a range of algorithmic
%optimizations. The former combined with the latter gives us the opportunity to
%strike a well-adjusted balance between model detail and computational workload,
%and thereby to improve upon the status quo of current implementations.

%Realistic animation and rendering of the ocean is an important aspect for simulators, movies and video games.
%By nature, the ocean is a difficult problem for Computer Graphics: it is a dynamic system, it combines wave trains
%at all scales, ranging from kilometric to millimetric. Worse, the ocean is usually viewed at several distances, from
%very close to the viewpoint to the horizon, increasing the multi-scale issue, and resulting in aliasing problems. The
%illumination comes from natural light sources (the Sun and the sky dome), is also dynamic, and often underlines
%the aliasing issues. In this paper, we present a new algorithm for modelling, animation, illumination and rendering
%of the ocean, in real-time, at all scales and for all viewing distances. Our algorithm is based on a hierarchical
%representation, combining geometry, normals and BRDF. For each viewing distance, we compute a simplified
%version of the geometry, and encode the missing details into the normal and the BRDF, depending on the level of
%detail required. We then use this hierarchical representation for illumination and rendering. Our algorithm runs
%in real-time, and produces highly realistic pictures and animations.

%Natural phenomena are a permanent topic of interest for computer graphics researchers and enthusiasts. To synthesise a believable
%depiction of an ocean surface is a complex task, especially in realtime rendering. The first part of our work will discuss the
%challenges posed by the task at hand, and discuss a variety of current approaches to the problem. The second part will present
%our give an
%overview of the underlying wave theory as well as the different oceanographic models employed in this work.