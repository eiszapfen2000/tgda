\chapter{Summary}
\label{ch:summary}

We stated in the introduction that in this work we seek to generate and render the
open ocean in real-time, which does not only look visually pleasing but also
believable to the observer.
Moreover, we chose to focus our efforts on wave geometry, because no lighting algorithm
is able to make good on unrealistic wave motion.
Wave energy spectrum from oceanographic research provides realistic wave geometry
and animation, given integral domain conversion is done correctly.
Discretisation is an issue, addressed by additional levels of detail.
Simple solution for tiling via sizes with irrational numbers.
Multi-resolution approach to reduce performance impact of level-of-detail
algorithm.
Ocean lighting scheme tailored to the above (Bruneton, Dupuy).

\section{Future Work}
The work presented in this thesis leaves room for improvement on several fronts.
One could take the next step in ocean surface synthesis and implement support
for wave energy spectra which are able to model shallow water, such as the
TMA spectrum \citep{Hughes:1984}. In that case, one would be required to
incorporate a dispersion relation which models both shallow and deep water,
including the necessary follow-up work to adapt the wave energy spectrum's
integral domain conversions to the new dispersion relation~\citep{Horvath:2015}.
%Moreover, one would have to adapt the wave energy integral scaling to match said dispersion relation.

Wave spectra aside, one could significantly improve upon performance by
following the lead of \cite{misc:oceanlightingfft} and generate all
pattern datasets on the GPU, as well as compute all \InvDiscreteFourierTransforms
on the GPU, while the static part of the underlying wave energy spectrum
could still be synthesized on the CPU. Moreover, one could look for a viable
alternative to rendering whitecaps, as the approach by \citet{article:whitecaps}
is computationally highly expensive with regard to pattern dataset synthesis.

%For performance reasons it would be highly beneficial to follow the lead of
%\cite{misc:oceanlightingfft} and generate all pattern datasets on the GPU,
%as well as compute all \InvDiscreteFourierTransforms on the GPU, while the
%underlying wave energy spectrum Equation~\ref{eq:dft_h0_k} could still be synthesized on the CPU.

Moreover, one could allow for more fine-grained control over pattern resolutions,
as it may be beneficial to configure resolutions for each dataset and each LOD
level separately.
