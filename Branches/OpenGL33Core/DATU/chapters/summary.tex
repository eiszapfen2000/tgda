\chapter{Summary}
\label{ch:summary}
%Paste the actual text from the introduction.
%We may approach a more specific subject, namely the synthesis of animated ocean
%surfaces which are both believable and computationally feasible.
%
%The scope of this thesis includes the generation, animation and rendering of the
%surface of an open ocean in real-time. We focus our interest on the synthesis of
%animated ocean surface geometry, for which we will adopt a set of models from
%oceanographic research. Specific properties of said models allow for
%easy addition and reduction of detail, as well as for a range of algorithmic
%optimizations. The former combined with the latter gives us the opportunity to
%strike a well-adjusted balance between model detail and computational workload,
%and thereby to improve upon the status quo of current implementations.
%
We stated in the introduction that in this work we seek to synthesize
animated ocean surfaces which are both believable and computationally feasible.
%We stated in the introduction that in this work we seek to synthesize the open
%ocean in such a manner that it does not only look believable to the observer,
%but is computationally feasible as well.
%
%We stated in the introduction that in this work we seek to generate and render the
%open ocean in real-time. Moreover, we aim for an ocean surface which does not only
%look visually pleasing, but also believable to the observer. Given that context,
%we chose to focus our efforts on wave geometry, because no lighting algorithm
%is able to make good on unrealistic wave shapes.
To achieve said objective, we picked the choppy wave algorithm by \citet{course:simulatingocean},
which adopts the wave spectrum concept from oceanographic research for
computer graphics purposes.
\citeauthor{course:simulatingocean} approximates the
perturbations produced by the water surface by sampling the wave
spectrum in frequency space via the \FastFourierTransform algorithm.
%
%we adopt the wave spectrum concept from
%oceanographic research for computer graphics purposes, as shown in
%\citet{course:simulatingocean}.
%
%The latter approximates the
%perturbations produced by the water surface by sampling the wave
%spectrum in frequency space via the \FastFourierTransform algorithm.
%For the purpose of ocean surface synthesis we opted for the approach by
%\citet{course:simulatingocean}, where the perturbations produced
%by the water surface are approximated by sampling the wave spectrum via
%the \FastFourierTransform algorithm.
The \FFT algorithm is highly efficient and has the additional advantage
of generating a perturbation pattern which seamlessly tiles the ocean surface.
A further upside of the wave spectrum approach is that oceanographic
research includes a wide range of wave spectra which are specifically
tailored to the deep water of the open ocean. Said wave spectra are
employed for oceanographic forecasts and simulations, therefore they
provide a high degree of realism.
Thus, we chose to implement four such wave energy spectra,
%\citep{article:PiersonMoskowitz1964, article:Hasselman1973, article:Donelan1985, article:Elfouhaily1997}, \citep{course:simulatingocean}
all well established in oceanographic literature, plus the wave
spectrum introduced by \citeauthor{course:simulatingocean} (Section~\ref{sec:wave_spectra}).
The latter spectrum was unable to produce results which meet the high
standard given by the other spectra, and therefore we dropped it
(Section \ref{sec:phillips_spectrum} and \ref{sec:results:synthesis}).
The remaining four spectra, on the other hand, allowed us to obtain
animated wave geometry at the level of quality we sought after,
but only after we had made sure that each wave spectrum had been converted
to the correct integral domain (Section~\ref{subsec:integral_domain_conversion}).
%

For the purpose of ocean surface lighting we decided to implement the
algorithm by \citet{misc:oceanlightingfft} as it gives believable results
and it has specifically been tailored to the wave spectrum of the open ocean.
After a successful implementation it turned out that only two of our
remaining wave spectra were able to unconditionally match the algorithms
stringent requirements, with a third spectrum keeping up under most
conditions (Section~\ref{sec:results:lighting}).
Furthermore, we implemented the work by \citet{article:whitecaps} to
incorporate whitecaps into the ocean surface's lighting model.

With ocean surface synthesis and lighting taken care of, we found that
as long as we employ only one perturbation pattern, we may not
be able to reproduce a high-quality ocean surface for all possible
viewing situations. Depending on the observer's viewpoint, we may be
faced with tiling artifacts or lack of detail.
Therefore we adopted the approach by~\citet{misc:oceanlightingfft},
which addresses both matters by superimposing multiple, complementary perturbation
patterns of different size (Section~\ref{sec:level_of_detail}).
With multiple patterns at hand, where for each pattern we have to
generate and Fourier transform up to nine separate spectra (one for vertical
perturbation, two for horizontal perturbation, two for gradients,
four for whitecaps), pattern generation had become computationally too
expensive to still allow for real-time operation.
Therefore we took a number of measures to restore real-time behaviour.
First, we split surface synthesis into two separate threads, where the first
one generates the pattern's actual spectra, and the second one
applies the \InvFourierTransform on said spectra (Section~\ref{sec:demo_application}).
%First, we moved pattern generation as well as 
%and transformation via \FFT to separate threads each (Section~\ref{sec:demo_application}).
Second, as the spectrum consists of a static part and a time-dependent part,
we made sure to compute only the time-dependent part of the spectrum for each
frame (Section~\ref{sec:spectrum_synthesis}).
Third, as all of our spectra are hermitian, we were able to compute the
\InvFourierTransform of two spectra at once (Section~\ref{sec:discrete_fourier_transform}).
%Hence,
%we were able to reduce the maximum number of \InvFourierTransforms per pattern
%from nine to five.
Fourth, we switched the \InvFourierTransform from double precision to single
precision, accelerating its computation by a factor of two
(Section~\ref{sec:performance:fft}).

With performance back to real-time framerates, we improved upon the work by
\citet{misc:oceanlightingfft} with a multi-resolution variant, where
the water surface's wave geometry is synthesized at a lower resolution
than its respective normal vectors and whitecaps
(Section~\ref{sec:implementation:multires}).
The lower resolution lead to a considerable speedup of the \FourierTransform
stage, but not to a significant degradation in visual quality for most scenes
(Section~\ref{sec:results:fidelity}).
%For a large range of distinct scenes the reduced resolution did not seem
%to degrade visual quality in a significant way.
Thus we decided to permanently keep the multi-resolution approach, as it
allows for an even more fine-grained balance between model detail and performance.
%

%Furthermore, we implemented the lighting algorithm by \cite{article:whitecaps},
%which has been specifically tailored to wave spectra.

%
%
%Thus we are given the
%means to further improve performance by 
%
%The latter approach
%allowed us for a more fine-grained balance between model detail and performance. 

%Naturally, multiple patterns imply an increase in computational cost of
%pattern generation. To still be able to generate the ocean surface in
%real-time
%
%With only one perturbation pattern at hand, one may be faced with tiling
%artifacts or lack of detail, depending on the observer's viewpoint.


%First, discretisation and multiple patterns, for proper sampling. Then,
%hermitian spectra to improve performance. Then multi-resolution, again,
%for performance.

%Ocean lighting scheme tailored to the above (Bruneton, Dupuy), another spectrum
%disqualified.

%With our goal being to render the open ocean, we found that discretisation
%of the wave spectrum had proven to be an issue, as we were not sampling enough
%distinct \wavenumbers to reproduce a high-quality ocean surface for all
%possible viewing situations. We addressed the matter by using multiple,
%complementary wave patterns which sample different, non-overlapping parts
%of the wave spectrum. During this work, up to four of such patterns where
%always sufficient.

%but only after we made sure to meet a set of preconditions. First and
%foremost, we had to ensure that each wave spectrum of interest is
%converted to the correct integral domain. Second,  
%
%Wave energy spectrum from oceanographic research provides realistic wave geometry
%and animation, given integral domain conversion is done correctly.
%Discretisation is an issue, addressed by additional levels of detail.
%Simple solution for tiling via sizes with irrational numbers.
%Multi-resolution approach to reduce performance impact of level-of-detail
%algorithm.
%Ocean lighting scheme tailored to the above (Bruneton, Dupuy).

\section{Future Work}
%The work presented in this thesis leaves room for improvement on several fronts.
Now that we have given a compact summary of our work, we may point out a few
potential improvements.
One could take the next step in ocean surface synthesis and implement support
for wave spectra which incorporate shallow water, such as the TMA spectrum
\citep{Hughes:1984}. In that case, one would be required to use a dispersion
relation which models both shallow and deep water, including the necessary
follow-up work to adapt the wave energy spectrum's integral domain conversions
to the new dispersion relation~\citep{Horvath:2015}.
%Moreover, one would have to adapt the wave energy integral scaling to match said dispersion relation.
Wave spectra aside, one could significantly improve upon performance by
following the lead of \cite{misc:oceanlightingfft} and generate all
surface patterns on the GPU, as well as compute all \FourierTransforms
on the GPU, while the static part of the underlying wave spectrum
would still be synthesized on the CPU. Additionally, it would beneficial
to look for a viable alternative to rendering whitecaps, as the approach
by \citet{article:whitecaps} is computationally highly expensive,
especially with regard to pattern synthesis.
%For performance reasons it would be highly beneficial to follow the lead of
%\cite{misc:oceanlightingfft} and generate all pattern datasets on the GPU,
%as well as compute all \InvDiscreteFourierTransforms on the GPU, while the
%underlying wave energy spectrum Equation~\ref{eq:dft_h0_k} could still be synthesized on the CPU.
%Last, to better balance performance and visual fidelity, one could
%allow for even more fine-grained control over pattern resolutions,
%as it may be of advantage to be able to configure the resolution for
%each spectrum per pattern.

\section{Concluding Remarks}
We have shown the integration of the wave spectrum concept into a
coherent framework for the real-time display of believable ocean
surfaces. Specific properties of the wave spectrum allow us, one,
to synthesize the water surface with high efficiency, and two, to
strike a well adjusted balance between model detail and performance.
%Furthermore, we have discussed the adequacy of a set of
%wave spectrum models for real-time rendering purposes. only 
%We further improved upon said balance
%by decoupling surface geometry detail from surface lighting detail.
%To improve upon existing implementations, we decoupled surface
%geometry resolution from surface lighting resolution, enabling us
%to achieve an even more fine-grained balance between model detail
%and performance.
%To improve upon existing implementations and achieve an even more
%fine-grained control over said balance, we succesfully decoupled
%surface geometry detail from surface lighting detail.
% 
% achieve 
%Additionally, we have discussed a set of optimizations that are
%specific to the wave spectrum and allow 
%
%
%BRAK
