\chapter{Summary}
\label{ch:summary}

We stated in the introduction that in this work we seek to generate and render the
open ocean in real-time. Moreover, we aim for an ocean surface which does not only
look visually pleasing, but also believable to the observer. Given that context,
we chose to focus our efforts on wave geometry, because no lighting algorithm
is able to make good on unrealistic wave shapes.

For the purpose of ocean surface synthesis we opted for the approach by
\citet{course:simulatingocean}, where the perturbations produced
by the water surface are approximated by sampling the wave spectrum via
the \FastFourierTransform algorithm. The \FFT algorithm is highly
efficient and has the additional advantage of generating a perturbation
pattern which is able to seamlessly tile the ocean surface.
An additional upside of the wave spectrum approach is that oceanographic
research includes a wide range of wave spectra which are specifically
tailored to the deep water of the open ocean. Said wave spectra are
employed for oceanographic forecasts and simulations, therefore they
provide a high degree of realism.
Thus, we chose to implement four such wave energy spectra
\citep{article:PiersonMoskowitz1964, article:Hasselman1973, article:Donelan1985, article:Elfouhaily1997},
all well established in oceanographic literature, plus one wave
spectrum which stems from computer graphics~\citep{course:simulatingocean}.
The latter spectrum was unable to produce results which meet the high
standard given by the other spectra, and therefore we dropped it
(Section \ref{sec:phillips_spectrum} and \ref{sec:results:synthesis}).
The remaining four spectra, on the other hand, allowed us to obtain
animated wave geometry at the level of quality we sought after,
but only after we had made sure that each wave spectrum had been converted
to the correct integral domain (Section~\ref{subsec:integral_domain_conversion}).
%

We found that with only one perturbation pattern at hand, we may not
sample enough distinct \wavenumbers to reproduce a high-quality ocean
surface for all possible viewing situations. Depending on the
observer's viewpoint, we may be faced with tiling artifacts or lack
of detail. Therefore we adopted the approach by~\citet{misc:oceanlightingfft},
which addresses the matter by using multiple, complementary perturbation
patterns which sample different, non-overlapping parts of the wave spectrum.

With only one perturbation pattern at hand, one may be faced with tiling
artifacts or lack of detail, depending on the observer's viewpoint.


First, discretisation and multiple patterns, for proper sampling. Then,
hermitian spectra to improve performance. Then multi-resolution, again,
for performance.

Ocean lighting scheme tailored to the above (Bruneton, Dupuy), another spectrum
disqualified.

With our goal being to render the open ocean, we found that discretisation
of the wave spectrum had proven to be an issue, as we were not sampling enough
distinct \wavenumbers to reproduce a high-quality ocean surface for all
possible viewing situations. We addressed the matter by using multiple,
complementary wave patterns which sample different, non-overlapping parts
of the wave spectrum. During this work, up to four of such patterns where
always sufficient.

%but only after we made sure to meet a set of preconditions. First and
%foremost, we had to ensure that each wave spectrum of interest is
%converted to the correct integral domain. Second,  

Wave energy spectrum from oceanographic research provides realistic wave geometry
and animation, given integral domain conversion is done correctly.
Discretisation is an issue, addressed by additional levels of detail.
Simple solution for tiling via sizes with irrational numbers.
Multi-resolution approach to reduce performance impact of level-of-detail
algorithm.
Ocean lighting scheme tailored to the above (Bruneton, Dupuy).

\section{Future Work}
The work presented in this thesis leaves room for improvement on several fronts.

One could take the next step in ocean surface synthesis and implement support
for wave energy spectra which are able to model shallow water, such as the
TMA spectrum \citep{Hughes:1984}. In that case, one would be required to
incorporate a dispersion relation which models both shallow and deep water,
including the necessary follow-up work to adapt the wave energy spectrum's
integral domain conversions to the new dispersion relation~\citep{Horvath:2015}.
%Moreover, one would have to adapt the wave energy integral scaling to match said dispersion relation.

Wave spectra aside, one could significantly improve upon performance by
following the lead of \cite{misc:oceanlightingfft} and generate all
pattern datasets on the GPU, as well as compute all \InvDiscreteFourierTransforms
on the GPU, while the static part of the underlying wave energy spectrum
would still be synthesized on the CPU. Additionally, it would beneficial
to look for a viable alternative to rendering whitecaps, as the approach
by \citet{article:whitecaps} is computationally highly expensive,
especially with regard to pattern dataset synthesis.

%For performance reasons it would be highly beneficial to follow the lead of
%\cite{misc:oceanlightingfft} and generate all pattern datasets on the GPU,
%as well as compute all \InvDiscreteFourierTransforms on the GPU, while the
%underlying wave energy spectrum Equation~\ref{eq:dft_h0_k} could still be synthesized on the CPU.

Last, to better balance performance and visual fidelity, one could allow
for more fine-grained control over pattern resolutions, as it may be
of advantage to be able to configure the resolution for each dataset for
each pattern separately.
