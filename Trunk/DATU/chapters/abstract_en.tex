\chapter*{Abstract}

%According to the guidelines of the faculty, an abstract in English has to be inserted here.
% Since the dawn of computer graphics natural phenomena have served as a permanent topic of interest for researchers and enthusiasts.
% To reproduce nature's elegance and beauty has proven to be a task of immense complexity. Skylight, clouds, a sunset. Water in various environments.
% Glass of water generating caustics. A child exploring fluid dynamics in the bathtub. A lake at rest, reflecting the surrounding mountains.
% The endless ocean bearing detail on every scale: the foam and spray caused by breaking waves, the glimmer on the ocean surface caused by waves
% with wavelengths smaller than a centimeter, the subtle shift in water color depending on the local underwater flora and fauna.

Natural phenomena are a permanent topic of interest for computer graphics researchers and enthusiasts. To synthesise a believable
depiction of an ocean surface is a complex task, especially in realtime rendering. The first part of our work will discuss the
challenges posed by the task at hand, and discuss a variety of current approaches to the problem. The second part will present
our give an
overview of the underlying wave theory as well as the different oceanographic models employed in this work.