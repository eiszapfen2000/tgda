\chapter{Summary}
\label{ch:summary}

\section{Performance}

Simple \InvFourierTransform vs \InvFourierTransform of two hermitian spectra
Pattern synthesis at full resolution for all datasets
Pattern synthesis with our multi-resolution approach

\section{Visual Fidelity}

Images of all partial terms involved in ocean lighting (specular by sun,
reflection of sky, refracted light, whitecaps)

Troubles with reflection vectors below horizon
Troubles with bright patches caused by reflection
Troubles with dark patches caused by reflection
Troubles with sundisk (XYZ to sRGB)

Tiling images

Comparison full resolution patterns vs multi-resolution scheme

Maybe different tonemapping settings

\section{Future Work}
The work presented in this thesis leaves room for improvement.

The next step in ocean surface synthesis based on wave energy spectra would be
to support spectra which are able to model shallow water, such as the
TMA spectrum \citep{Hughes:1984}.
Thus, one would be required to incorporate a dispersion relation which models
shallow as well as deep water. Moreover, one would have
to adapt the wave energy integral scaling to match said dispersion relation.
\cite{Horvath:2015}

For performance reasons it would be highly beneficial to follow the lead of
\cite{misc:oceanlightingfft} and generate all pattern datasets on the GPU,
as well as compute all \InvDiscreteFourierTransforms on the GPU, while the
underlying wave energy spectrum Equation~\ref{eq:dft_h0_k} could still be synthesized on the CPU.

Moreover, one could allow for more fine-grained control over pattern resolutions,
as it may be beneficial to configure resolutions for each dataset and each LOD
level separately.
