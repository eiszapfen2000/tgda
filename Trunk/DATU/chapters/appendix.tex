\chapter{Appendix}
\section{Skydome Irradiance}
%
\begin{figure}
\centering
\def\svgwidth{0.5\textwidth}
\input{figures/hemispherical-coordinates.pdf_tex}
\caption{The location of point $x$ is defined by the hemispherical coordinate
pair $(\varphi, \theta)$, where $\varphi$ is called the azimuthal angle, and
$\theta$ the polar angle. We assume the radius of the hemisphere
is equal one.}
\label{fig:hemispherical:coordinates}
\end{figure}
%
Irradiance is defined as the radiant flux \emph{received} by a surface per unit
area. Note that we do not employ radiometric units in this work, because it
would exceed the scope of this thesis. We employ CIE XYZ \citep{ciexyz:1931}
coordinates instead.
CIE XYZ is a linear space and is therefore suitable for the task at hand,
namely numerical integration.

We assume the skylight is infinitely far away, and no occlusion takes place.
It follows that the irradiance generated by the skylight is equal at all
positions in the scene.
Thus, one needs to compute the irradiance only when the Preetham sky's
parameters change.

We may obtain the irradiance of the skydome by integrating the
Preetham skylight model over the entire hemisphere. Fortunately, both the
Preetham skylight, and the irradiance integral are defined in terms of
hemispherical coordinates. Figure~\ref{fig:hemispherical:coordinates} depicts
the core concept of hemispherical coordinates, where the coordinate axes match
the ones employed throughout this work. We may compute the irradiance $E$ as
follows:
\begin{equation}
E = \int_{0}^{2\pi}\int_{0}^{\pi/2} P(\varphi,\theta) \sin{\theta}\mathrm{d}\theta\mathrm{d}\varphi
\end{equation}
where $P$ denotes the Preetham skylight contribution in terms of CIE XYZ coordinates.

%
%
\section{Mean Square Slope}
\label{app:mss}
%
The mean square slopes of the ocean surface give a measure of the
statistical distribution of the surface waves' slopes~\citep{Massel:2011}.
The ocean lighting algorithm by \cite{misc:oceanlightingfft} requires us to compute
the mean square slopes of the wave energy spectrum $\Theta(\mvec{k})$.
Let $\mvec{k} = (k_x, k_z)$ be the \wavevector,
then the mean square slope in the upwind direction, $mss_x$, and the mean square
slope in the crosswind direction, $mss_z$, are defined as follows:
\begin{align}
mss_x = \iint_{\mvec{k}} k_x^2 \Theta(\mvec{k}) \mathrm{d}\mvec{k} &&
mss_z = \iint_{\mvec{k}} k_z^2 \Theta(\mvec{k}) \mathrm{d}\mvec{k}
\end{align}
%
The total mean square, on the other hand, is independent of direction
and defined as follows:
\begin{equation}
mss = mss_x + mss_z = \iint_{\mvec{k}} \norm{\mvec{k}}^2 \Theta(\mvec{k}) \mathrm{d}\mvec{k} = \int_{0}^{\infty}k^2 \Theta(k)\mathrm{d}k
\end{equation}