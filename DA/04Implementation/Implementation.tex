\chapter{Implementation}

This chapter discusses all software modules implemented during the
development of the demo application, as well as the design decisions made in
the process.

\begin{itemize}
 \item \textit{NPEngine}, a realtime rendering framework
 \item \textit{Ocean Bakery}, a preprocessing tool which bakes all
necessary ocean data
 \item \textit{Demo Application}, shaders an ocean
\end{itemize}

\section{NPEngine}

For the implementation of the Demo application a lightweight realtime
rendering framework has been developed from scratch. It is written in
\textit{Objective-C} and uses \textit{GNUstep} as basis.

\subsection{Objective-C}

Objective-C has been designed as an object-oriented extension to the
\textit{C} programming language. As such it is a very thin layer on top of, as
well as a strict superset of C. As a result Objective-C can freely include C
code within it's classes, and the Objective-C compiler is able to compile any
existing C code. 
Five reasons that make Objective-C most suiting for the task at hand:

\begin{itemize}
 \item \textit{Messages}
 \item \textit{Protocols}
 \item \textit{Categories}
 \item \textit{Dynamic typing}
 \item \textit{Reflection}
\end{itemize}

\subsubsection{Messages}
Objective-C inherits it's object-oriented model from Smalltalk, which is based
on sending messages to object instances. There is a fundamental difference
between calling a method and sending a message. The former implies more or less
hardwiring between a method name and a section of code at compile time, while
the latter remains a name which is resolved at runtime. Basically it is the
receiving object's job to interpret the messages which get sent to it. As a
consequence the message passing system has no type checking: the receiving
object is not guaranteed to be able to respond to a certain message; in the
case it does not, it ignores the message and returns a null pointer.

\lstset{language=[Objective]C, backgroundcolor=\color{LightViolet}}
\begin{lstlisting}[captionpos=b, caption=An Objective-C message send,
label=lst_objcmsgsend]{}
[ object method:parameter ];
\end{lstlisting}

\subsubsection{Protocols}
Protocols are similar to \textit{Java}\cite{book:java-language} and
\textit{C\#}\cite{book:csharp-language} \textit{Interfaces}. A protocol defines
a set of methods which any class can declare itself to implement,
listing~\ref{lst_objcprotocol} shows a simple example. Protocols, like
Interfaces, are a neat way to decouple functionality from the class hierarchy.

\lstset{language=[Objective]C, backgroundcolor=\color{LightViolet}}
\begin{lstlisting}[captionpos=b, caption=An Objective-C protocol and a
class adopting it.,
label=lst_objcprotocol]{}
@protocol MyProtocol
- (id) aMethod:(id)aParameter;
- (id) anotherMethod:(id)anotherParameter;
@end

@interface MyClass : NSObject < MyProtocol >
@end
\end{lstlisting}

\subsubsection{Categories}
Categories make it possible to distribute the implementation of a single class
to different files. Therefore the developer can group related methods into a
category to achieve better readability. For instance, one could create a
\textit{Splitting} category on the \textit{NSString} class, collecting all of
the string splitting methods in one place.
Furthermore a categories methods are added to a class at runtime, thus
removing the need to recompile that class or even have access to its source
code. By using categories the developer is given a tool to extend existing
classes without resorting to modify the original class or to create a subclass.
Moreover, due to the runtime resolve of categories, it is possible to overwrite
the existing methods of a class. Category methods are generally preferred over
the original methods, providing a way to modify existing classes. Categories are
only able to add new methods to a class, but not new instance variables.
Listing~\ref{lst_objccategory} shows how to define a simple category.

\lstset{language=[Objective]C, backgroundcolor=\color{LightViolet}}
\begin{lstlisting}[captionpos=b, caption=An Objective-C category added to the
NSString class., label=lst_objccategory]{}
@interface NSString (MyCategory)
- (id) appendInteger:(int)anInteger;
@end
\end{lstlisting}

Categories also make it possible to attach protocols to a class outside of its
main interface, listing~\ref{lst_objccategoryprotocol} shows a minimal example.

\lstset{language=[Objective]C, backgroundcolor=\color{LightViolet}}
\begin{lstlisting}[captionpos=b, caption=Attaching a protocol to an existing
class by the means of a category., label=lst_objccategoryprotocol]{}

@protocol CategoryProtocol
- (id) aMethod:(id)aParameter;
@end

@interface MyClass (MyCategory) < CategoryProtcol >
@end

\end{lstlisting}

\subsubsection{Dynamic typing}
Objective-C, like Smalltalk, makes heavy use of dynamic typing: every message
can be sent to every object, without even knowing if the receiving object is
able to respond; the message is not interpreted until runtime anyway, because
of the dynamic dispatching mechanism. This provides the means for some
interesting design decisions in \textit{Foundation} and \textit{AppKit}, which
are discussed later. \FIXME{Add references to sections.}
Objective-C makes heavy use of the \textit{id} type. Simply put, id is a pointer
to an object, no matter what the actual type of the object pointed to is. By
using id the compiler can not make any assumptions about the object pointed to,
meaning the compiler does not know to which messages the object pointed to is
acutally able to respond. In that case all responsibility lies with the runtime.

On the other hand, the developer may add static type information to variables,
this information is then checked at compile time. It is the developer's task to
find a sane balance between untyped and typed information, since both have their
assets and drawbacks. Listing~\ref{lst_objctypesnoinfo},
\ref{lst_objctypesprotocolinfo} and \ref{lst_objctypesclasslinfo} show the same
method with increasing type information for it's argument.

\lstset{language=[Objective]C, backgroundcolor=\color{LightViolet}}
\begin{lstlisting}[captionpos=b, caption=Argument without type information.,
label=lst_objctypesnoinfo]{}
- (void) setValue:(id)newValue;
\end{lstlisting}

\lstset{language=[Objective]C, backgroundcolor=\color{LightViolet}}
\begin{lstlisting}[captionpos=b, caption=Argument with protocol type
information.,
label=lst_objctypesprotocolinfo]{}
- (void) setValue:(id <SomeProtocol>)newValue;
\end{lstlisting}

\lstset{language=[Objective]C, backgroundcolor=\color{LightViolet}}
\begin{lstlisting}[captionpos=b, caption=Argument with class type information.,
label=lst_objctypesclasslinfo]{}
- (void) setValue:(NSNumber *)newValue;
\end{lstlisting}

\subsubsection{Reflection}
The Objective-C runtime provides support for some reflective features: objects
can be asked about their class, the methods they implement, the protocols they
implement, and even what member variables they have. Those features are made
possible due every object having a pointer to the \textit{Class} object it was
created by. \textit{Class} objects are structs holding all information about an
implemented class, like member variables, methods and protocols. It is even
possible to use those class objects in actual code,
listing~\ref{lst_objcclassfromstring} shows a simple example.

\lstset{language=[Objective]C, backgroundcolor=\color{LightViolet},
morecomment=[s]{@"}{"}}
\begin{lstlisting}[captionpos=b, caption=Creating an object of a class given
only the class name., label=lst_objcclassfromstring, morekeywords={Class}]{}
    Class myClass = NSClassFromString(@"MyClass");
    id myObject = [[ myClass alloc ] init ];
\end{lstlisting}

\lstset{language=[Objective]C, backgroundcolor=\color{LightViolet},
commentstyle=\color{DarkGreen}}
\begin{lstlisting}[captionpos=b, caption=Check if an object responds to a
specific message.,
label=lst_objcinstancerespindstoselector]{}
if ( [ myObject respondsToSelector:@selector(myMethod:) ] == YES )
{
    // do something
}
\end{lstlisting}

\lstset{language=[Objective]C, backgroundcolor=\color{LightViolet},
commentstyle=\color{DarkGreen}}
\begin{lstlisting}[captionpos=b, caption=Check if an object implements a
specific protocol.,
label=lst_objcinstanceimplementsprotocol]{}
if ( [ myObject conformsToProtocol:@protocol(MyProtocol) ] == YES )
{
    // do something
}
\end{lstlisting}

Actually getting information about member variables is more difficult, one has
to dive deeper in the runtime to be able to retrieve it. There are rare cases
where this kind of object introspection is really needed, the most prominent
being the user interface loading mechanism in AppKit, where one objects needs
to be connected to a member variable of another object, but no setter method is
available. Another use case the author can think of, would be a scripting
mechnism. One could probably get away by just constructing message identifiers
(selectors) using strings, on the other hand direct member variable access
through the use of strings would imply less message sending overhead.

\subsection{GNUstep}
GNUstep is an open source implementation of the OpenStep
specification\cite{misc:OpenStepSpec} by NeXT Computer Inc.\cite{misc:NeXT}.
OpenStep is based heavily on Objective-C, placing two separate libraries at
the developers disposal: the \textit{Foundation} and the \textit{AppKit}. In
1996 Apple bought NeXT and based it's new operating system Mac OS X, which has
been released in 2001, on the OpenStep specification. What once was called
OpenStep is now known by the name Cocoa. Cocoa has undergone major improvements
during the further development of Mac OS X, which as of now has reached version
10.6. GNUstep is an open-source implementation of OpenStep which runs on a
number of different platforms, including the major Unices and Windows. Since
the release of Mac OS X the GNUstep project seeks to implement most of the
extensions Apple made to the original OpenStep.

\subsubsection{Foundation}

The foundation contains a set of non-graphical Objective-C classes which deal
with strings, containers, date and time, distributed objects, networking,
threading,  interthread and interprocess communication, file management and
more.

Collections, that is what containers are called in Foundation, have a rather
unusual property: they do not have to be homogenous, meaning that one instance
of a collection can not hold just one type of objects, it can hold everything
as long as it is an object. Listing~\ref{lst_objcheterocontainer} shows an
example. The downside is, the developer can not use these collections for
integral types such as int or float.

\lstset{language=[Objective]C, backgroundcolor=\color{LightViolet},
commentstyle=\color{DarkGreen}}
\begin{lstlisting}[captionpos=b, caption=Argument with class type information.,
label=lst_objcheterocontainer]{}
NSArray * array = [NSArray arrayWithObjects:[NSDate
date], [NSData data], @"MyString", nil];
\end{lstlisting}

Due the heterogeneous property of collections the developer is not forced to
implement a common superclass for all objects he wants to put in one concrete
instance of a container anymore.

\subsubsection{AppKit}

The AppKit provides classes aimed at the development of graphical applications.
The most interesting concept of AppKit is that it does not represent graphical
user interfaces as code. Instead, a graphical user interface created in
Interface Builder/Gorm is saved as a bunch of serialised objects and their
connections to each other. There are two distinct connection types:
\begin{itemize}
 \item \textit{Outlets} assign a GUI element to a member variable of a
controlling object instance.
 \item \textit{Actions} do consist of a \textit{target} and a message which
gets sent to the target.
\end{itemize}

Typically at program startup the file containing the GUI gets loaded, meaning
that the objects contained are deserialised and afterwards all connections are
setup using names of member variables and messages in conjunction with the
Objective-C reflection system. In the case a message or a member variable can
not be found, related connections are simply not setup, but all others are.

\subsection{OpenGL}
OpenGL, short for \textit{Open Graphics Library}, is a specification defining a
platform-independent 3D application programming interface. OpenGL was first
released to the public in 1992 by SGI, as a successor to the company's
internally developed IrisGL. OpenGL is designed to standardise access to
graphics hardware, pushing the development responsibility for hardware interface
programs, also known as device drivers, to hardware manufacturers. After the
release of OpenGL to the public, SGI led the creation of the OpenGL
architectural review board (OpenGL ARB)\cite{misc:opengl-arb}, an independent
consortium consisting of a group of companies that would maintain and expand the
OpenGL standard. In 2006 further development of the OpenGL standard has been
taken over by the \textit{KhronosGroup} \cite{misc:opengl-khronos}, bringing
significant advancements to the OpenGL standard since then.

OpenGL consists of two separate parts, the platform-independent core API and a
platform-dependent layer. The core API gives programmers access to all
rendering pipeline stages described in Section \ref{RenderingPipelineS}. The
platform-dependent layer establishes a connection between an OpenGL
\textit{rendering context} and the operating system's actual windowing system.
Prominent examples for such an platform-dependent OpenGL layer are \textit{WGL}
on Microsoft Windows and \textit{GLX} on the X Windowing System.

OpenGL is designed as an extensible API. Each vendor who provides an actual
OpenGL implementation, be it in software or hardware accelerated, is given the
means to augment his implementation of the OpenGL standard through the use of
OpenGL \textit{extensions}. Extensions as such are addenda to the OpenGL
specification, usually implementing new features not yet included in the OpenGL
core. A list of all OpenGL extensions and their textual additions to the
standard are to be found in the OpenGL \textit{extension registry}
\cite{misc:opengl-registry}. Each new release of the OpenGL core specification
incorporates a set of extensions in order to integrate new functionality. At the
time of writing, the current version of the OpenGL standard was 3.2.

\subsection{Other Libraries}

In addition to GNUstep as class library and OpenGL for rendering, NPEngine
integrates several other libraries and frameworks either at engine or at
application level.

\subsubsection{DevIL}

The \textit{Developer's Image Library}\cite{misc:devil} allows for loading and
storing of a large set of image file formats, qualifiying it as an adequate
tool for texture loading. Furthermore \textit{DevIL} provides an OpenGL like
interface, hence upvalueing it's ease of use in the context of a framework
mostly based on OpenGL.

\subsubsection{OpenAL}

\textit{NPEngine}'s sound system uses \textit{OpenAL}~\cite{misc:openal}, a free
cross-platform 3D audio API for playing sound, maintained by
\textit{Creative}~\cite{misc:creative}. As the name may suggest, OpenAL's
programming interface is similar to the one of OpenGL. OpenAL also inherits
the implementation philosophy behind OpenGL: the specification has to be
implemented in it's entirety, but it does not matter which parts run in software
or are hardware accelerated. Furthermore the concept of extensions has been
adopted too, making it easy for vendors to implement new features.

Creative does provide an OpenAL sample implementation, which at the time of
writing is already deprecated and replaced by either \textit{OpenAL
Soft}~\cite{misc:openal-soft} or vendor specific implementations. In contrast to
some vendor implementations of OpenAL, OpenAL Soft is a pure software
implementation unable to take advantage of any hardware acceleration. On the
other hand it does implement most of the existing extensions, hence making
developers less dependent on specific (hardware, implementation) combinations.

\subsubsection{Ogg Vorbis}

The \textit{Ogg Vorbis}~\cite{misc:ogg-vorbis} library is used to decode Ogg
sound files. Vorbis is an open, patent- and royalty-free, compressed audio
codec, for lossy audio compression on par with \textit{MPEG-4
AAC}\cite{misc:mpeg-aac}~\cite{misc:mpeg-aac-standard} regarding quality and
compression ratio. Ogg is the associated multimedia container format which may
encapsulates the Vorbis data stream as well as other data streams such as
\textit{Theora}\cite{misc:ogg-theora}. The NPEngine sound system uses Ogg Vorbis
files both as fixed-size samples and as continuous streams for playing audio.

\subsubsection{NVIDIA CgFX}

\textit{NVIDIA CgFX}\cite{misc:nvcg} is a toolkit which places a high
level programming interface for GPU shader programs at the programmer's
disposal. CgFX is similar in functionality to the \textit{DirectX Effect
Framework}\cite{misc:directx}\cite{book:effect-hlsl}, as it gives the
shader programmer the means to group different vertex, geometry and fragment
programs to so called \textit{techniques}. A technique consists of one or more
\textit{pass(es)}, each of which is made up of an optional set of state
assignments and a set of programs. CgFX's goal is to reduce complexity on the
implementation side and add it to the shader authoring side. One can create
rather complex rendering passes consisting of various vertex, geometry and
fragment programs as well as render state changes just by writing a CgFX shader.

There are a number of downsides to CgFX, the most disturbing one of them is that
CgFX takes away a lot of control from the actual programmer, especially texture
unit management is completely taken over by CgFX. Additionally CgFX is known to
be overly CPU intensive and generally prone to regressions regarding performance
and feature completeness. Moreover CgFX is a library maintained by NVIDIA,
so there is no guarantee that CgFX will work as well on rivaling hardware as on
their own.

\subsubsection{PRNG}

\textit{PRNG}\cite{misc:prng} is a collection of state-of-the-art random number
generators implemented as a highly portable C library. The library is used to
generate gaussian random numbers for all wave spectra related computations in
the preprocessing tool.

\subsubsection{FFTW}
The \textit{Fastest Fourier Transform in the West} library\cite{misc:fftw} is an
open source C library for computing the discrete Fourier transform in one or
more dimensions. The preprocessing tool employs FFTW to transform the wave
spectra from frequency domain to spatial domain.

\section{Projected Grid}
Viewing Pipeline
Projected Grid

\section{R2VB}
Render to VertexBuffer

\section{Lighting}
Reflection
Refraction

\section{Animation}
2D Textures
3D Texture
CUDA

\section{Color management}
color profiles
gamma
srgb sampler/framebuffer